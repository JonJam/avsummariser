
% Glossary entries for use within the document.
% http://en.wikibooks.org/wiki/LaTeX/Glossary
% Glossary entries must be referenced before they are used.

% See Justification.tex for example usage.

% The acronyms listed are not put down as references. An acronym can contain text
% From a glossary entry, but the latex/makeglossaries pair in Makefile need to be run TWICE
% Acronyms can be used the same as a glossary entry via \gls{acronym-key}
\newacronym{IDEacr}{IDE}{Integrated Development Environment}

% Glossary entries are put into the glossary as long as they are cited using \gls{key} \Gls{key} or \GLS{key} (the variations just affect the capitolisation of the name printed at the point of reference.)
\newglossaryentry{IDE}{name=Integrated Development Environment,
description={A piece of software which}}

\newacronym{LaTexacr}{LaTex}{}

\newglossaryentry{LaTex}{name = , 
description ={A form of document production designed for techinical
publications. The format allows for the simpler production of papers which
contain mathmatical forumla and varying layouts and referencing formats,
using a more programming like approach.}}

\newacronym{BBCacr}{BBC}{British Broadcasting Corperation}

\newglossaryentry{BBC}{name = The British Broadcasting Corperation,
description = {A tax payer paid for television and radio based broadcasting
house. The BBC aims to provide entertainment and knowledge to the British
isles, our remaining empire and the rest of the world through the world
service. The BBC is known worldwide for its impartial news stories and has a
reputation as a trustworthy, quality news source.}}

\newacronym{IAMacr}{IAM}{The Intelligence, Agents, Multimedia Group}

\newglossaryentry{IAM}{name = The Intelligence, Agents, Multimedia Group,
description ={A group of researchers baased at Southampton University. These
researchers look into behavious of agents, intelligence and multimedia in
various novel and interesting applications throughout the spectrum of
technology and science. They are multidisciplinary and interdisciplinary based
allowing for world class research to be conducted into the creation and
application of such systems.}}

\newacronym{OpenIMAJacr}{OpenIMAJ}{Open Intelligent Multimedia Analysis in Java
toolkit}

\newglossaryentry{OpenIMAJ}{name = Open Intelligent Multimedia Analysis in Java
toolkit, description ={A software library written in Java with the goal of
assisting the analysis of audiovisual data. OpenIMAJ contains a number of
useful methods for most multimedia processing needs. It was created by IAM at
Southampton university during 2011 and recently presented to the wider
computer science community during a conference in November 2011.}}

