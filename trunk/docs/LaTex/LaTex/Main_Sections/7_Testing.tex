\newpage
\section{Code \& UI Testing}

In accordance with our previously decided testing strategy (see Section
\ref{TestingStyles}) these following sections detail those tests the team have
performed upon the project as it has been developed and once it was completed. These tests comprise of unit tests, system tests and user goal testing. A complete list of these tests can be found in Appendix \ref{sec:UnitTestAppendix}.

\subsection{Unit Testing}
These tests ensure that the smallest modules of code work as intended. Shown here
are some of those tests carried out by the team as the system was being developed. Most of these tests are over
conditional sections of code such as if checks and for loops, which are essential
to the author's intended functionality of the code. All of these test locations are prefixed by: uk.ecs.gdp.avsummariser. 

Below is a table detailing ten of the most significant unit tests. The rest can be found in Appendix \ref{sec:TestingAppendix} in Section \ref{sec:UnitTestAppendix}.

\begin{center}
\begin{tabular}{| p{18pt} | p{90pt}| p{78pt}| p{75pt} | p{62pt} | p{35pt} |}
\hline
Test & Test location & Test description & Expected result & Actual result & Success\\\hline
1 & model.audio.\newline FrequencyDetector & For loop, does it exit when it should & Exit after ten iterations & Exits after ten iterations & Y\\\hline
2 & model.audio.\newline VolumeDetector & In the method
mergeLoudSectionsInVideo will the if check for the current selection being null
work as intended & Enter an input of null and the if loop should be entered & If
loop entered & Y\\\hline
3 & model.subtitles. \newline PersonNameFinder & In the method
findPersonNamesInSubtitlesUsingTVDB are all the Strings in the ArrayList looked
at through the for loop & Should loop through the for loop for each String in
the ArrayList, so enter ten Strings and the for loop shall loop ten times &
Enters the for loop ten times & Y\\ \hline
4 & model.summary. \newline ExportSummary & In the method exportSummary there
is an if check which checks to ensure that all a file extension is acceptable & All
permitted file extensions should pass & Only permitted file extensions pass the
check & Y\\ \hline
\end{tabular}
\end{center}

\begin{center}
\begin{tabular}{| p{18pt} | p{90pt}| p{78pt}| p{75pt} | p{62pt} | p{35pt} |}
\hline
5 & model.section. \newline comparator. \newline LocationMentioned \newline Comparator & In the method compare the if and else if check area was 
examined to ensure the correct region was entered for the entered input & First score is bigger than the second, should enter the first region
 & First region is entered & Y\\ \hline
6 & model.section. \newline comparator. \newline LocationMentioned \newline Comparator & In the method compare the if and else if check area was 
examined to ensure the correct region was entered for the entered input & First score is smaller than the second, should enter the first region
 & First region is entered & Y\\ \hline
7 & model.tvdb. \newline SeriesSearch & In the method searchSeries enter the if region only when the user has entered a 
string & Enter nothing & Does not enter the if region, enters the else region instead & Y\\ \hline
8 & model.subtitles. \newline LocationName \newline Finder & In the
method getAllLocationNames we test to see if the for loop iterates through all
of the Strings in the ArrayList & Enter ten Strings and count that the for loop
iterates ten times & The for loop iterated ten times & Y\\ \hline
9 & model.subtitles. \newline NameFinder & In the method
setUpSentenceAndTokenModels we test to see whether the IOException try catch
section works as intended & When an IOException occurs a stack trace should be
printed out & A stack trace appears when an IOException occurs & Y\\ \hline
10 & model.subtitles. \newline NameFinder & In the method
setUpSentenceAndTokenModels we test to see whether the FileNotFoundException try
catch section works as intended & When a FileNotFoundException occurs a stack
trace should be printed out & A stack trace appears when a FileNotFoundException
occurs & Y\\ \hline
\end{tabular}
\end{center}

\subsection{System Testing}
These tests examine how the system as a whole works. These following tests are a subset of those already
performed upon the team's project. Others tests may be found in Appendix \ref{sec:TestingAppendix} in Section \ref{sec:SystemTestAppendix}

\begin{center}
\begin{tabular}{| p{18pt} | p{92pt}| p{68pt}| p{152pt} | p{44pt} |}
\hline
Test &Test														&Input				&Expected Outcome					&Actual Outcome
\\\hline
1	 &Load Video (Multiple video files)									&MP4 \newline video files			&Loaded Successfully						&Y			
\\\hline	
2	&Select video \newline (Non selected \newline previously)								&Video file \newline without subs	&Update display to show video selected and processing for video begins.		&Y
\\\hline	
3	&Output tabs \newline update correctly \newline (Whilst not \newline viewing)						&Video file \newline with subs &Output Tabs updated Successfully whilst viewing.	&Y					
\\\hline						
4	&Generate summary (Using metrics: \newline genre, trailer type and duration)			&Video file + \newline  Subs. Top \newline Gear series \newline selected.		&Summary generated Successfully using metrics&Y
\\\hline						
5	&Generate summary \newline (Multiple videos \newline with mix of Sub\newline  files and none,\newline TVDB info)	&2 Video files. \newline 1 + subs. \newline 1 without \newline TVDB info.	&Summary generated Successfully.	&Y
\\\hline					
6	&Play summary \newline (Then play another video/summary)					&1 Video file with subs. \newline TVDB info \newline loaded. 	&Video plays Successfully.		&Y	
\\\hline								
7	&Export summary\newline (1 video without\newline Subs, no TVDB \newline info)	&Video file \newline without Subs. No TVDB \newline info loaded.	&Export summary Successfully. However due to development of this feature on Windows, on different operating systems, export directory plus file name to be exported always had Windows ``\textbackslash" file path seperator.			&N
\\\hline
8	&RETEST OF 7											&											&Export summary Successfully. Null pointer exception thrown due to Map for people \& location being empty per video.&N
\\\hline
9	&RETEST OF 7											&											&Export summary Successfully.	&Y
\\\hline			
10	&Export summary again after complete (Checking Summary object in model has updated path)	&Video file \newline without Subs. No TVDB \newline info.	&Export summary Successfully and moved trailer succesfully.	&Y	
\\\hline	
\end{tabular}
\end{center}

\subsection{User's goals}

These tests intend to look at how well the user's goals have been met by the
team's finished system. Due to time constraints upon the team there was no time to prepare for a suite of user acceptance tests or for the testing to be
carried out, as the team had initially planned. These tests aim to simulate in some ways
how the user might interact with the system and to see whether it meets their
expectations.

\begin{center}
\begin{longtable}{| p{20pt} | p{105pt}| p{105pt} | p{105pt} | p{35pt} |}
\hline
Test & Test description & Expected result & Actual result & Success\\\hline
1 & The ability to enter a range of different audiovisual formats & A number of
different formats may be entered such as .ts and .mp4 & Only .mp4 is accepted &
N\\\hline
2 & Detection of shots within a video & The same number of shots should be
detected by the system as a trained editor would find within a video, a team
member found 347 in a short section of video & The system finds approximately
the same number of shots as the team members do, the number it found was 361.
This is mostly like down to the teams inexperience with video editing & Y\\\hline
3 & Detecting the points at which people appear in a video & Through
facedetection and recognition the system should be able to find and identify
faces in a video selection and record the moments these faces appear & Each
detected character has all the points in which they appeared on screen recorded
and skippable to & Y\\\hline
4 & Detecting those points at which characters are mentioned in the subtitles &
Should record the time stamps of each occasion in which a character is
mentioned in the subtitles & Each character has a list of those moments in which
they were mentioned in the subtitles & Y\\\hline
5 & Detecting the points at which locations are mentioned in the subtitles & A
list of points when a location is mentioned in the subtitles should be created &
An accurate list of those moments in which locations are mentioned is created &
Y\\\hline
6 & Detecting loud portions of audio & Audio levels should be recorded and loud
portions can be used & Loud portions can be used as they are recorded down
using their timestamps & Y\\\hline
7 & Detecting music in the audio & The system should be able to read in some
audio and locate those sections which contain music & The system can not
separate music from other audio & N\\\hline
8 & Key shot identification & Shots which are key to the video should be
selected in some manner & Key shots are highlighted to allow the viewer to know
these shots are deemed important in some manner & Y\\\hline
9 & Identify the main characters & Those characters which would qualify as main
should be shown by the system & Those characters which appear frequently enough in a video are shown by the system to be main characters & Y\\\hline
10 & Identify the main locations & Those locations which are regarded to be main
should be shown & Those locations which appear frequently enough to be regarded
as main locations are shown & Y\\\hline
11 & Identify music & A given piece of audio should be able to be identified as
a named track by a certain band & Audio samples of original music may be
identified successfully as to the name of the song and the musicians who played
it & Y\\\hline
12 & Summarise the input files based on user selected metrics & A summary or
``trailer'' should be produced for the entered TV shows, such a trailer should
be concise, but also entertaining and placing an emphasise on the users chosen
metric & A summary is created for any input audiovisual data and these summaries
vary dependent upon the metrics chosen by the user & Y\\\hline
13 & Exporting a summary & A XML document should be produced to allow the
summary details to be viewed else where and for the summary to be reproduced &
A XML file is produced for a summary when desired & Y\\\hline
14 & Use OpenIMAJ & Using OpenIMAJ instead of recreating functionality & OpenIMAJ is used where possible & N
\\\hline
\end{longtable}
\end{center}