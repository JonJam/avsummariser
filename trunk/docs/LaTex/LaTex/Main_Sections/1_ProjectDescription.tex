\section{Project Description}

\subsection{Introduction}

The British Broadcasting Corporation (BBC) has a huge archive of millions of hours of audiovisual data which they have broadcasted over the past eighty four years. It is very difficult to effectively manage an archive of that size and it would be unfeasible to attempt to process all of the data manually. There is great potential to re-use this material in a multitude of ways such as building new programme material from existing content or reselling and re-broadcasting past programmes. However, currently any manipulation and analysis of this data has to be performed manually as there is no automated system in place to aid the BBC archiving department. These obviously cost both time and money, therefore the archived material is unable to be used to its full potential.

Such an issue costs the BBC a significant amount of resources and the automation of these functions should enable the archiving department to not only to save a significant amount of resources, but to fully utilise the material they have available. The real world issues involved with the project and the non-trivial nature of cataloging audiovisual data is what motivates the team to set goals such as the full automation of these processes.

\subsection{Project Problem}
\label{sec:ProjectProblem}
The system is primarily aimed at people working within the BBC who need to summarise their archived content in order to do activities 
such as intellectual property (IP) rights assessment or finding unbroadcast footage of specific programmes. However, it is also aimed at 
potential customers when the BBC want to “sell” a television series to another company. The team's customer has provided us with three 
potential scenarios that they wish us to consider when working out the systems goals. The provided potential scenarios are as follows:

\begin{itemize}
	\item{\textbf{Archive Monetisation} - The BBC wish to provide short summaries of items/collections from their archived content to prospective buyers. Unfortunately the BBC has limited resources and may not have the time to effectively create summaries for any given set of videos from their collection, an automated process though could potentially prove cheaper in terms of resources and more effective. Summaries need to represent all main elements of the video such as main characters and main action sequences.}
	\item{\textbf{Rights Assessment} - In order to be able to reuse some of their archived content the BBC need to be able to identify which Intellectual Property rights (e.g for musical pieces, particular actors) need to be cleared before they can do this. However, whilst the BBC have subtitles for their content they do not necessarily include the actors names, so they would like to be able to produce a summary of who is featured in the video content so that they can clear the IP rights accordingly.}
	\item{\textbf{Archive Content Selection} - The BBC may want to use some of their unbroadcasted archived content to produce extended versions of episodes or films. An automatic analyser of the content that is able to identify the range of topics/locations that made it into the final cut of the programme would allow an extended edition to be simply created by matching the unseen content appropriately.}
\end{itemize}

The team has decided to examine all of these scenarios as there are obvious similarities between them. An analysis of all scenarios is intended to produce as a comprehensive system as possible in the available time.

\subsection{Project Goals}
\label{sec:projectGoals}
The project’s goals were produced through the use of the provided project brief (Appendix \ref{sec:ProjBrief}) and the initial meeting with the team's client. The project goals which the team hopes to achieve are as follows:

\begin{itemize}
	\item{The ability to accept a range of different audiovisual formats as input.}
	\item{The ability to detect shots in video.}
	\item{The ability to detect when people appear in video.}
	\item{The ability to detect where people are mentioned in subtitles/audio.}
	\item{The ability to detect where locations are mentioned in subtitles/audio.}
	\item{The ability to detect loud sections in audio.}
	\item{The ability to detect music segments in audio.}
	\item{The ability to identify key shots in video.}
	\item{The ability to identify main characters.}
	\item{The ability to identify main locations.}
	\item{The ability to identify music segments.}
	\item{The ability to summarise input files using user selected metrics.}
	\item{The ability to export summary data produced.}
	\item{The use of IAM's library OpenIMAJ.}
\end{itemize}
		
\subsection{Final Project Scope}

The scope of the team's project is to deliver a polished, user friendly piece of
software that may be used easily by a non-technical client. The project
shall incorporate all the initial goals shown above and any new potential goals that emerge throughout the project. 

