 \section*{Abstract}

The BBC has an ever increasing archive of broadcast audiovisual footage that dates back over fifty years, 
which is stored for the purposes of content preservation and ensuring that the BBC's high standards are maintained. 
There is limited information attached to their archive causing the contained data to be difficult to summarise, hard to effectively classify and 
reuse. The lack of effective automated methods to allow for 
such summarisation costs the BBC a large amount of resources due to the manual processing involved, which is time consuming and allows mistakes.

In order to assist with the solution, the team's aim is to develop a piece of software which allows users to process and summarise a 
collection of material by applying video, audio and natural language processing techniques to produce a summary as well as XML 
files containing information about the collection based on a user’s criteria.

The report details all of the aspects of the project, such as how the team tackled the problem together, an in depth 
analysis of the software and the team's performance, as well as discussions upon the future work which could be done as an extension to the project.
